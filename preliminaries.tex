\section{Preliminaries}

\noindent
\textbf{Notation.}
Given a sequence $Y$, we address it using $Y[i]$ to mean the $i^\text{th}$ element (starting from $0$).
Negative indices address elements from the end, so $Y[-i]$ is the $i^\text{th}$ element from
the end, and $Y[-1]$ in particular is the last. We use $Y[i{:}j]$ to denote the subarray of $Y$
consisting of the elements indexed from $i$ (inclusive) to $j$ (exclusive). The notation $Y[i{:}]$ means the
subarray of $Y$ from $i$ onwards, while $Y[{:}j]$ means the subsequence of $Y$ up to (but not including) $j$.
We use $|Y|$ to denote the length of $Y$.
We say that $X$ is a, not necessarily strict,
\emph{prefix} of $Y$, denoted $X \preceq Y$
or $Y \succeq X$,
when every element of $X$ appears in $Y$ at the
same location (but $Y$ may have more elements beyond $|X|$).
We write $X[i] = \bot$ for $i \geq |X|$ to indicate that the sequence
has been exhausted.

\noindent
\textbf{Distributed Ledger Protocols.}
In this work, we are concerned with the composition of Distributed Ledger
Protocols (DLPs). A DLP $\Pi$ is an interactive machine (in modern programming
lingo corresponding to an OOP \emph{class}). The same
protocol $\Pi$ is \emph{instanciated} into multiple instances,
each called a \emph{party} (conceptually running on a different computer each).
We are interested in a population of parties, some of which are designated
as \emph{honest}, while others are designated as \emph{adversarial}.
All honest parties run the prescribed protocol $\Pi$, whereas adversarial
parties may run any protocol of their choice (including $\Pi$).
All adversarial parties are controlled by one colluding adversary $\mathcal{A}$.

\noindent
\textbf{Time and networking.}
We model time as taking place in discrete \emph{rounds} $1, 2, \ldots$.
During each round, all parties are allowed to run some (finite) computation.
The parties are allowed to communicate with one another through an
underlying \emph{network}. This is accomplished when the party \emph{broadcasts}
a message to the network. The network is external to the parties and ensures
every message broadcast by an honest party is delivered to every other honest
party. The adversary may delay the delivery of honest messages up to the
\emph{network delay} $\Delta$ rounds, which is known to the parties, but cannot delay
messages indefinitely. The adversary may also introduce new messages, reorder
existing messages, and do so differently for each receiving honest
party~\cite{backbone}.

\noindent
\textbf{Ledgers.}
A DLP offers two functionalities to the user running
it: \emph{write} and \emph{read}. The \emph{write} functionality allows the
user to \emph{write a transaction} $\tx$ to the DLP. The \emph{read}
functionality allows the user to \emph{read a ledger} $\Ledger$ from the DLP.
The ledger is a finite sequence of transactions
$\tx_1, \tx_2, \ldots, \tx_{|\Ledger|}$,
each annotated with a timestamp $r_1, r_2, \ldots, r_{|\Ledger|} \in \mathbb{N}$:

\[
  \Ledger = ((\tx_1, r_1), (\tx_2, r_2), \ldots, (\tx_{|\Ledger|}, r_{|\Ledger|}))
\]

The user of a DLP is typically a computer program such as a wallet.
We denote by $\Ledger^P_r$ the output of the \emph{read} functionality of
party $P$ when executed at the end of round $r$.

\noindent
\textbf{Security.}
Because $\Pi$ can be randomized, we are interested in particular
\emph{executions} of DLPs. An execution captures
everything that happened between the honest parties running the
protocol and the adversary, including the local state of each
party and the network messages that were exchanged. The execution
concludes after a finite amount of time. It is desirable that our
DLPs produce executions that satisfy the following virtues.

\begin{definition}[Safe]
  An execution $E$ of a DLP $\Pi$ is \emph{safe} if for all rounds $r_1, r_2$
  and all honest parties $P_1, P_2$ we have that
  $\Ledger^{P_1}_{r_1} \preceq \Ledger^{P_2}_{r_2}$ or
  $\Ledger^{P_1}_{r_1} \succeq \Ledger^{P_2}_{r_2}$.
\end{definition}

Without loss of generality, we assume that all \emph{safe} protocols
are also \emph{sticky}: The ledger of an honest party never decreases in
length. Any non-sticky safe DLP can easily be made sticky~\cite{streamlet}.

\begin{definition}[Live]
  An execution $E$ of a DLP $\Pi$ is \emph{live} with parameter $u \in \mathbb{N}$ if, whenever
  \emph{all} honest parties attempt to write the transaction $\tx$ during
  rounds $r, r + 1, \ldots, r + u - 1$, the transaction appears in all
  honest ledgers at all rounds $r' \geq r + u$.
\end{definition}

The above definition requires that \emph{all} honest parties attempt to
write a transaction during rounds $r, \ldots, r + u - 1$. However, we
assume, without loss of generality,
that parties \emph{gossip} any transaction they wish to write onto
the underlying ledger. All honest parties that receive a gossiped transaction
will also attempt to include it, and, so, if \emph{one} honest party
attempts to introduce a transaction at round $r$, the transaction will
make it to everyone's ledger by round $r + \Delta + u$.

\begin{definition}[Timely]\label{def:timely}
  An execution $E$ of a DLP $\Pi$ is \emph{timely} with parameter $v \in \mathbb{N}$
  if for all honest parties $P$
  and rounds $r_1 < r_2$ it holds that:

  \begin{enumerate}
    \item The timestamps in $\Ledger^P_{r_1}$ are non-decreasing.\label{def:timely-increasing}
    \item The timestamp of $\Ledger^P_{r_1}[-1]$ is at most $r_1$.\label{def:timely-past}
    \item The timestamps of $\Ledger^P_{r_2}[|\Ledger^P_{r_1}|{:}]$ are at least $r_1 - v$.\label{def:timely-chunk}
  \end{enumerate}
\end{definition}

All popular blockchain protocols report timestamps together with their
transactions and ensure their timeliness. For example, Bitcoin and Ethereum
produce blocks each of which contains a timestamp. These timestamps can be copied
into the transactions therein. Because the respective protocols do not accept
chains with decreasing timestamps, or blocks with future timestamps the timeliness
points~\ref{def:timely-increasing} and~\ref{def:timely-past} are ensured.
Point~\ref{def:timely-chunk} is more subtle and asks that very old transactions will
not suddenly appear in the ledger.

\begin{definition}[Secure]
  An execution $E$ of a DLP $\Pi$ is \emph{secure} (with parameters $(u, v)$) if the execution
  is \emph{safe}, \emph{timely} (with parameter $v$), and \emph{live} (with parameter $u$).
\end{definition}

\noindent
\textbf{SMT protocol.}
Definition of SMT ...
