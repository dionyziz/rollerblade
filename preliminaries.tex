\section{Preliminaries}

\noindent
\textbf{Notation.}
Given a sequence $Y$, we address it using $Y[i]$ to mean the $i^\text{th}$ element (starting from $0$).
Negative indices address elements from the end, so $Y[-i]$ is the $i^\text{th}$ element from
the end, and $Y[-1]$ in particular is the last. We use $Y[i{:}j]$ to denote the subarray of $Y$
consisting of the elements indexed from $i$ (inclusive) to $j$ (exclusive). The notation $Y[i{:}]$ means the
subarray of $Y$ from $i$ onwards, while $Y[{:}j]$ means the subsequence of $Y$ up to (but not including) $j$.
We use $|Y|$ to denote the length of $Y$.
We say that $X$ is a, not necessarily strict,
\emph{prefix} of $Y$, denoted $X \preceq Y$
or $Y \succeq X$,
when every element of $X$ appears in $Y$ at the
same location (but $Y$ may have more elements beyond $|X|$).
We write $X[i] = \bot$ for $i \geq |X|$ to indicate that the sequence
has been exhausted.
We use $[n]$ to denote the set $\{1, \ldots, n\}$.

\noindent
\textbf{Distributed Protocols.}
In this work, we are concerned with the composition of distributed
protocols.
% TODO: Maybe move this to separate paragraph outside of distributed protocols

A distributed protocol $\Pi$ is an interactive Turing machine (ITM).
To simplify the exposition, we treat this ITM
as an object-oriented class $\Pi$ and the respective \emph{party}
(a particular ITM \emph{instance}) as an object (class instance).

\begin{definition}[Distributed Protocol]
  A \emph{distributed protocol} is an ITM which exposes the follow methods:

  \begin{itemize}
    \item $\construct^\net()$:
          This method called when the protocol is \emph{instantiated} into a
          \emph{party}. We denote this using the notation $\new \Pi^\net()$,
          which returns a party that can be interacted with.
          It is also given oracle access to a network functionality
          $\net$.
    \item $\lwrite(\data)$:
          Takes user input by accepting some \emph{data} string.
    \item $\lread()$:
          Produces user output in the form of some \emph{data} string.
          We mandate that,
          upon its completion, the execution of a \emph{read} instruction
          does not change the state of the interactive machine.
    \item $\execute()$:
          Executes a single round of the protocol.
          Within $\execute$, the machine can
          use the \emph{netout} functionality of $\net$ to send messages
          (as many times as it wishes)
          and the \emph{netin} functionality of $\net$ to receive messages
          (once per round).
  \end{itemize}
\end{definition}

The same protocol $\Pi$ is \emph{instantiated} into multiple instances,
each called a \emph{party} (conceptually running on a different computer each).
We are interested in a population of parties, some of which are designated
as \emph{honest}, while others are designated as \emph{adversarial}.
All honest parties run the prescribed protocol $\Pi$, whereas adversarial
parties may run any protocol of their choice (including $\Pi$).
All adversarial parties are controlled by one colluding adversary $\mathcal{A}$.

% TODO: Write out the environment?
\begin{definition}[Permissioned Distributed Protocol]
  A distributed protocol is \emph{permissioned} if its \emph{construct}
  method accepts its own identity $j \in \mathbb{N}$
  and the number $n \in \mathbb{N}$ of parties it will coordinate with.
\end{definition}

\noindent
\textbf{Executions.}
Because $\Pi$ can be randomized, we are interested in particular
\emph{executions} $\Exec$ of protocols. An execution captures
everything that happened between the honest parties running the
protocol and the adversary, including the local state of each
party and the network messages that were exchanged. The execution
concludes after a finite amount of time.

% TODO: Consider moving time above the "distributed protocol" definition,
% since the concept of "round" is used within that definition.
\noindent
\textbf{Time.}
We model time as taking place in discrete \emph{rounds} $1, 2, \ldots$.
The parties run in \emph{lockstep}:
During each round, each party is allowed to run some (finite) computation.
Each party is initialized (by having its $\construct$ function called)
before round $0$ commences.
Subsequently, round $0$ starts by having the $\execute$ function of each
party called, without any inputs provided by the environment (in the form
of ``writes'' or network inputs).
By the end of each round, the machine can produce some network outputs
to the environment. During every non-zero round, the environment
makes some network inputs available to the party being executed,
subject to the network constraints defined below.

\noindent
\textbf{Network.}
The parties are allowed to communicate with one another through an
underlying \emph{network}. We make use of two types of networks.

% TODO: this is probably useless, since all its guarantees are provided through
% the underlying liveness
% TODO: citation for gossip
\begin{definition}[Gossip Network]
  A \emph{gossip network}
  allows any party to \emph{diffuse} a message $m$ to the rest of the parties.
  The network ensures that, if the sender is honest, then every other honest party
  will receive the diffused message $m$.
\end{definition}

% TODO: citation for authenticated channels
\begin{definition}[Authenticated Channels]
  An \emph{authenticated channels network}
  allows any party $P$ to \emph{point-to-point} send a message to another
  party $Q$. The network faithfully reports the sending party to the receiving
  party, and the adversary cannot forge the origin of messages.
\end{definition}

% TODO: consider the psync setting?
\begin{definition}[Synchrony]
  A network is \emph{synchronous} with parameter $\Delta \in \mathbb{N}$ if
  any message $m$ sent by an honest party $P$
  at the end of round $r$ is delivered by the beginning of round $r + \Delta$.
\end{definition}

In the gossip network model, the adversary may introduce an arbitrary number of
messages and forge their origin, and may also reorder messages before they are
delivered to the honest parties. In the authenticated channels model, the adversary
is not allowed to forge the origin of messages, but may send an arbitrary number
of messages from each of the corrupted parties she controls.
In both models, the adversary may send different messages
to each honest party at each round. The delay of each honestly sent message is
also adversarially controlled, as long as the bound $\Delta$ is respected.
However, the adversary is not allowed to censor honestly produced messages
after the bound $\Delta$ has elapsed.

% TODO: Consider the case where no authenticated channels are available?
% authenticated channels --> transcriptions
% signatures --> certificates
% Observation: certificates are too strong for authenticated channels

\noindent
\textbf{Ledgers.}
% TODO: Do we need non-temporal ledgers?
\begin{definition}[Temporal Ledger]
A distributed protocol $\Pi$, together with a \emph{transaction validity language} $\mathcal{V}$
is a \emph{temporal ledger} protocol if its \emph{read} and \emph{write}
functionalities have the following semantics:

\begin{itemize}
  \item $\lwrite(\tx)$: The write functionality accepts a \emph{transaction}, which is
        a string that belongs to $\mathcal{V}$.
  \item $L \gets \lread()$: The read functionality returns a \emph{ledger} $L \in (\mathbb{N} \times \mathcal{V})^*$, which is
        a finite sequence of pairs $(r, \tx)$ where $\tx \in \mathcal{V}$ is a transaction, and
        $r$ is a \emph{round} indicating the time at which the transaction in question
        is recorded on the ledger.
\end{itemize}
\end{definition}

It is desirable that our
ledger protocols produce executions that satisfy the following virtues.
Let $\Ledger[P][][r]$ denote the ledger reported by the honest party $P$
issuing a \emph{read} instruction to its ledger protocol at round $r$.

\begin{definition}[Safe]
  An execution $\Exec$ of a ledger protocol $\Pi$ is \emph{safe} if for all rounds $r_1, r_2 \in \mathbb{N}$
  and all honest parties $P_1, P_2$ running instances $\Pi^1, \Pi^2$,
  we have that
  $\Ledger[P_1][][r_1] \preceq \Ledger[P_2][][r_2]$ or
  $\Ledger[P_1][][r_1] \succeq \Ledger[P_2][][r_2]$.
\end{definition}

% TODO: get rid of this stickiness
% Without loss of generality, we assume that all \emph{safe} protocols
% are also \emph{sticky}: The ledger of an honest party never decreases in
% length. Any non-sticky safe ledger protocol can be easily made sticky~\cite{streamlet}.

\begin{definition}[Live]
  An execution $\Exec$ of a ledger protocol $\Pi$ is \emph{live} with parameter $u \in \mathbb{N}$ if, whenever
  \emph{all} honest parties attempt to write the transaction $\tx$ during
  rounds $r, r + 1, \ldots, r + u - 1$, the transaction appears in all
  honest ledgers at all rounds $r' \geq r + u$.
\end{definition}

The above definition requires that \emph{all} honest parties attempt to
write a transaction during rounds $r, \ldots, r + u - 1$. However, we
assume, without loss of generality,
that parties \emph{gossip} any transaction they wish to write onto
the underlying ledger. All honest parties that receive a gossiped transaction
will also attempt to include it, and, so, if \emph{one} honest party
attempts to introduce a transaction at round $r$, the transaction will
make it to everyone's ledger by round $r + \Delta + u$.

The following definition is first introduced in this work, as it will prove
immensely useful for composability, but is a natural property that all well-designed
blockchain systems have:

\begin{definition}[Timely]\label{def:timely}
  An execution $\Exec$ of a ledger protocol $\Pi$ is \emph{timely} with parameter $v \in \mathbb{N}$
  if for all honest parties $P$ and rounds $r_1 \leq r_2$ it holds that:

  \begin{enumerate}
    \item The rounds recorded in $\Ledger[P][][r_1]$ are non-decreasing.\label{def:timely-increasing}
    \item The round of $\Ledger[P][][r_1][-1]$ is at most $r_1$.\label{def:timely-past}
    \item The rounds recorded in $\Ledger[P][][r_2][|\Ledger[P][][r_1]|{:}]$ are at least $r_1 - v$.\label{def:timely-chunk}
  \end{enumerate}
\end{definition}

All popular blockchain protocols report timestamps together with their
transactions and ensure their timeliness. For example, Bitcoin and Ethereum
produce blocks each of which contains a timestamp. These timestamps can be copied
into the transactions therein when \emph{reading}. Because the respective protocols do not accept
chains with decreasing timestamps, or blocks with future timestamps the timeliness
points~\ref{def:timely-increasing} and~\ref{def:timely-past} are ensured.
Point~\ref{def:timely-chunk} is more subtle and asks that very old transactions will
not suddenly appear in the ledger.
% TODO: prove timeliness of Bitcoin?

\begin{definition}[Secure]
  An execution $\Exec$ of a ledger protocol $\Pi$ is \emph{secure} (with parameters $(u, v)$) if the execution
  is \emph{safe}, \emph{timely} (with parameter $v$), and \emph{live} (with parameter $u$).
\end{definition}
