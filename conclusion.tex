\section{Conclusion}

\myparagraph[Summary]
We have developed a generic construction that allows running a replicated emulation
of (virtually) any distributed protocol using underlying ledgers as the communication
mechanism. During the process of proving our construction secure, we developed a
technical framework to discuss the analogy between the party setting and the emulated
setting, and proved that the two settings are equivalent.
Our main security result took the form of a simulation-based argument in two main theorems:
The Emulation Theorem (showing that, within the view of a single compositor, the party
setting and the emulated setting are equivalent), and the Replication Theorem (showing that
different compositors share the same view within the same execution). Our construction
can be used for various applications, for example to run a consensus protocol on top
of existing distributed ledgers, giving rise to a layer 2 rollup that provides better
security guarantees than any of the underlying ledger protocols alone.

\myparagraph[Related work]
Building reliable systems out of unreliable components
is a classical problem~\cite{von1956probabilistic,moore1956reliable}.
In consensus, Lamport's Byzantine Fault Tolerance problem~\cite{shostak1982byzantine}
aims to solve a reliability problem, where different processors disagree about their
outcomes. The composition of multiple \emph{blockchain} protocols was explored by
\emph{Fitzi et al.}~\cite{combiners}, but for the purpose of performance in terms of
\emph{latency}, not reliability. In their paper, they also introduce the notion of
\emph{relative persistence}, in which they talk of \emph{dynamic ledgers} (\emph{cf.}, our
\emph{temporal ledgers}) and transaction \emph{ranks} (\emph{cf.}, our \emph{recorded rounds})
which is related, but not equivalent, to our notion of \emph{timeliness}.
They also define the notion of a \emph{blockchain combiner} (\emph{cf.}, our \emph{compositors}).
Their protocol is \emph{passive} (\emph{cf.} TrustBoost~\cite{trustboost}),
meaning the ``combiner'' does not achieve full consensus (\cite[Section 5]{combiners}).
Their ``combiner'' is an instance of our \emph{compositors}, which
further allow for generic distributed protocols to run in an emulated and
replicated fashion, and achieve full consensus.

The idea of borrowing security has been explored in \emph{merged mining}~\cite{namecoin},
\emph{merged staking}~\cite{pos-sidechains}, and
\emph{checkpointing}~\cite{karakostas2021securing},
The idea of composing ledgers to achieve a more reliable overlay ledger
was first proposed in a short Cosmos GitHub issue called
\emph{recursive Tendermint}~\cite{recursive-tendermint}.
This concept was expanded upon by TrustBoost~\cite{trustboost}
where they build a composition using Cosmos as the underlying
construction, IBC for cross-chain communication, and Information Theoretic
HotStuff as the overlay protocol. They conjecture that their construction
is secure. However, they stop short of proving the security of their construction.
Their security theorem in the so-called ``active mode'' (\cite[Theorem 2]{trustboost})
states the variable $m$ in the theorem statement. That variable is
interpreted in the \emph{party setting} in the \emph{proof}, but in the \emph{emulated setting}
in the rest of the paper. Therefore, the theorem's positive or negative statement
for $m$ interpreted as ledgers (and not parties) is not proven.
The correspondence between the party setting and the emulated setting is
only conjectured in the short paragraph ``Security guarantee''~\cite[Section 4.1]{trustboost}.
Proving this correspondence requires significant technical work and the framework
which we develop in this paper.
In the present work, we
answer the question TrustBoost left open affirmative and calculate the correct
parametrization to instantiate their system securely (e.g., the calculation
of $\Delta$ in Theorem~\ref{conj:emulation}). We note that any secure
deployment of TrustBoost must include a correct choice of the parameter $\Delta$,
whose calculation is missing from their paper.
