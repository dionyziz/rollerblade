\section{Definitions}\label{sec:definitions}

\begin{definition}[Temporal Ledger]
  A distributed protocol $\Pi$, together with a \emph{transaction validity language} $\mathcal{V}$
  is a \emph{temporal ledger} protocol if its \emph{read} and \emph{write}
  functionalities have the following semantics:

  \begin{itemize}
    \item $\lwrite(\tx)$: The write functionality accepts a \emph{transaction}, which is
          a string that belongs to $\mathcal{V}$.
    \item $L \gets \lread()$: The read functionality returns a \emph{ledger} $L \in (\mathbb{N} \times \mathcal{V})^*$, which is
          a finite sequence of pairs $(r, \tx)$ where $\tx \in \mathcal{V}$ is a transaction, and
          $r$ is a \emph{round} indicating the time at which the transaction in question
          is recorded on the ledger.
  \end{itemize}
\end{definition}

The following definition is first introduced in this work,
as it will prove
immensely useful for composability, but is a natural property that all well-designed
blockchain systems have:

\begin{definition}[Timely]\label{def:timely}
  An execution $\Exec$ of a temporal ledger protocol $\Pi$ is \emph{timely} with parameter $v \in \mathbb{N}$
  if for all honest parties $P$ and rounds $r_1$ it holds that:

  \begin{enumerate}
    \item The rounds recorded in $\Ledger[P][][r_1]$ are non-decreasing.\label{def:timely-increasing}
    \item The round recorded at $\Ledger[P][][r_1][-1]$ is at most $r_1$.\label{def:timely-past}
    \item For all $r_2 \geq r_1$, the rounds recorded in $\Ledger[P][][r_2][|\Ledger[P][][r_1]|{:}]$ are
          newer than $r_1 - v$.\label{def:timely-chunk}
  \end{enumerate}
\end{definition}

All popular blockchain protocols report timestamps together with their
transactions and ensure their timeliness. For example, Bitcoin and Ethereum
produce blocks each of which contains a timestamp. These timestamps can be copied
into the transactions therein when \emph{reading}. Because the respective protocols do not accept
chains with decreasing timestamps, or blocks with future timestamps the timeliness
points~\ref{def:timely-increasing} and~\ref{def:timely-past} are ensured.
Point~\ref{def:timely-chunk} is more subtle and asks that very old transactions will
not suddenly appear in the ledger.

\iflong
As an example protocol, we include a proof that Bitcoin is timely in
Appendix~\ref{sec:appendix}.
\fi

\begin{definition}[Temporal Ledger Security]
  An execution $\Exec$ of a temporal ledger protocol $\Pi$ is \emph{secure} (with parameters $(u, v)$) if the execution
  is \emph{safe}, \emph{timely} (with parameter $v$), and \emph{live} (with parameter $u$).
\end{definition}

\subsection{The Setting}

We are given $n \in \mathbb{N}$ ledger protocols
$\Y[][1], \Y[][2], \ldots, \Y[][i], \ldots, \Y[][n]$,
the so-called \emph{underlying} ledger protocols.
While mathematically, these $\Y$s are interactive Turing machines of ledger protocols,
in practice, these are preexisting, already operational ledger protocol executions
such as Bitcoin, Ethereum, and Cardano, for which we have access to already running full nodes
and we are asked to compose on top of.

We are also given a distributed protocol $\Pi$ (not necessarily a ledger protocol),
the so-called \emph{overlay} protocol. We will emulate an $n$-party execution of $\Pi$,
with each $i$ of these $n$ parties corresponding to the underlying ledger protocol $\Y[][i]$.

The users of the protocol are $m \in \mathbb{N}$ \rollerblade \emph{clients} termed
$\RB[1], \RB[2], \ldots, \RB[j], \ldots, \RB[m]$ (with, potentially $m \neq n$).
\emph{Each} $\RB[j]$ client runs a separate full node $\Y[j][1], \ldots, \Y[j][i], \ldots, \Y[j][n]$
for each of the underlying $\Y$s. The $\RB$ nodes do not have direct network communication, but only
use the read/write functionalities of their respective $\Y$ instances to communicate.
For example, when party $\RB[1]$ \emph{writes} a transaction $\tx$ to its $\Y[1][1]$ instance,
this transaction will eventually appear in $\RB[2]$'s $\Y[2][1]$ instance ledger output,
as long as $\Y[][1]$ is live.
We will use $\Ledger[j][i][r] \gets \Y[j][i][r].\lread()$ to refer to the ledger reported at
round $r$ by the full node instance $\Y[j][i]$ running the underlying ledger protocol $\Y[][i]$
operated by the overlay party $\RB[j]$
(this ledger, like all ledgers, is a sequence of round/transaction pairs, and its
$k^\text{th}$ round/transaction pair is $\Ledger[j][i][r][k]$).

%
% TODO: move this to consensus section
% Our goal is to build a \emph{new} DLP $\RB$
% on top of these such that, if the majority of \emph{underlying}
% $\wheel[1], \ldots, \wheel[n]$ are \emph{secure}, then so is the
% \emph{overlay} DLP $\RB$. As $\RB$ is a DLP, it will have its own population of $m$ nodes
% $\RB_1, \RB_2, \ldots, \RB_m$. The $\RB$ DLP will have its own type of transaction and a
% ledger consisting of these transactions. For example, it can maintain its own coin\footnote{
% We will not be concerned with bridging this coin from the overlay ledger to the underlying
% ledgers. This can be done using various standard bridging techniques which are orthogonal
% to this work~\cite{pow-sidechains,pos-sidechains,zkbridge}.}. Importantly, similarly to
% a rollup, \emph{we will not make any additional security assumptions about the honesty of the
% $\RB$ nodes}.
%

We now describe the requirements of our underlying and overlay protocols.

\subsection{Rollerblade Underlying Requirements}

Firstly, in order to record data on our underlying ledgers, we require that any arbitrary string
can be written to them. This is called a \emph{bulletin}.

\myparagraph[Bulletins]
% We do not want to modify the underlying DLPs to support our protocol. However, the underlying DLPs
% do not understand the transaction semantics of our overlay ledger. We will therefore use the underlying
% DLPs as a simple lazy transaction ordering service. We will assume that we can take any
% arbitrary string $w$ and \emph{encode} it into a transaction suitable for each underlying ledger.
% Such transactions are \emph{always} accepted by the underlying ledger, and their contents
% are not validated (beyond checking that a minimum fee is suitably paid to avoid spamming attacks).
% We call a DLP that supports writing such arbitrary data into it a \emph{bulletin}.
A bulletin ledger protocol offers two additional
functions $\encode$ and $\decode$:
$\tx \gets \textsf{encode}(s)$ and $s \gets \textsf{decode}(\tx)$.
The $\textsf{encode}$ function takes a string $s$ and encodes it into a transaction $\tx$ that can be \emph{written}
into the ledger and is guaranteed to be accepted.
The $\textsf{decode}$ function takes a transaction $\tx$
and, if it is a bulletin transaction, decodes it back into $s$.
Otherwise, $\textsf{decode}$ can return $\bot$.
All transactions produced by $\textsf{encode}$
are bulletin transactions, but the adversary can also introduce arbitrary bulletin transactions of her choice
indiscriminately. The ledger may also include non-bulletin transactions among the bulletin transactions.

\begin{definition}[Bulletin Board]
  A ledger protocol $\Pi$ accompanied by a pair of computable
  functions $(\encode, \decode)$, of which $\decode$ is deterministic,
  is called a \emph{bulletin board} if it holds that, for any $s \in \{0, 1\}^*$,
  the output of $\tx = \encode(s)$ is always a valid
  transaction and that $\decode(\encode(s)) = s$.
\end{definition}

Bulletins provide ordering and data availability of arbitrary data without checking
any semantic validity. As such, they constitute a \emph{lazy} use of a ledger~\cite{lazyledger,lazylight}.
All popular blockchains such as, for example, Ethereum and Bitcoin are bulletins.
Bitcoin allows the recording of arbitrary data using \textsf{OP\_RETURN}
transactions, whereas Ethereum allows such recording by including the
data in the \textsf{CALLDATA} of a smart contract call, or in the parameters
of an event.

% TODO: generalize from certificates to transcriptions
% (and talk about adversarial advantage in the round in which the adversary produced the transcription)
% TODO: clean up this definition
\begin{definition}[Certifiability]
  A ledger protocol $\Pi$ accompanied by a computable functionality $\transcribe$
  and a computable deterministic (non-interactive) function $\untranscribe$ is called
  \emph{certifiable}. The functionality $\transcribe$ is called on a full node
  and returns a \emph{transcription} of its current ledger. The function
  $\untranscribe$ is called with the transcription as the parameter,
  and is hoped to return the original ledger.
  The certifiable protocol is \emph{live}
  % TODO: Can we unify this definition of liveness/safety with the ledger protocol
  % using one definition for both?
  if, in addition to the liveness requirements of the ledger protocol,
  whenever $\transcribe()$ is called on an honest
  party $P$ with a ledger $L$ and returns a transcription $\tau$,
  then $\untranscribe(\tau) = L$.

  % TODO: I don't like that this talks about "whenever" untranscribe is called...
  % it might be called incorrectly? Also this seems kind of internal to the construction...
  The certifiable protocol is \emph{safe}
  if,
  in addition to the safety requirements of the ledger protocol,
  whenever an honest party $P_1$ executes $\untranscribe(\tau)$ at round $r_1$
  with some (honestly or adversarially produced) transcription $\tau$,
  then for all honest parties $P_2$ at round $r_2$, it holds that
  $\Ledger[P_1][][r_1] \preceq \Ledger[P_2][][r_2] \lor \Ledger[P_2][][r_2] \preceq \Ledger[P_1][][r_1]$.
\end{definition}

\subsection{Compositors}

To aid the analysis, we assume that the ITIs contain all the previous transcript
of their execution (a history of machine configurations).

% \begin{definition}[Determinism]
%       Let $\Pi^\Sigma$ be a State Machine Replication protocol that makes
%       oracle use of a signature protocol $\Sigma = (\textsf{Gen}, \textsf{Sig}, \textsf{Ver})$.
%       The protocol $\Pi$ is called \emph{soft deterministic} if,
%       when executed as an interactive turing machine $\Pi^\Sigma(i)$ initialized for party
%       with index $i$, the machine does not make use of any local randomness.
% \end{definition}
%
% TODO: define underlying, overlay, good, VIEW...

% TODO: Move this to definitions? Expand this intuitive explanation,
% and give it a title
% Potential generalization: Use a different H_Z and H_Y that indicate
% which Ys are "good" and which Zs are supposed to be "honest", and
% make H_Z a function of H_Y, to be defined by the DPCP.
% This would allow having a different number of Zs and Ys,
% and each of the Zs could depend on different, and potentially multiple, Ys;
% and each Y could help multiple Zs.

A \emph{compositor} is a protocol that
runs an \emph{overlay} distributed protocol $\Pi$ on top of a set of
$n$ \emph{underlying} distributed protocols $\Y[][1], \ldots, \Y[][i], \ldots, \Y[][n]$.
The protocol $\Pi$ is generally designed
to work in the \emph{party setting} with a network (such as an authenticated channels network)
connecting its instances directly.
The compositor $\Lambda$ executes
$\Pi$ in the \emph{emulated setting} by utilizing the underlying
$\Y[][i]$ protocols to help with the emulation and communication between the
overlay protocol instances.
Multiple instances $\LLambda[1], \ldots, \LLambda[j], \ldots, \LLambda[m]$
of the compositor are executed.
Each $j^\text{th}$ compositor instance promises to emulate the execution of multiple
instances of $\Pi$ ($\Z[j][1], \ldots, \Z[j][i], \ldots \Z[j][n]$).
These emulated $\Z$ instances should behave similarly to instances of $\Pi$
running in a stand-alone setting.

\begin{definition}[Compositor]
  A \emph{compositor} $\Lambda$ with overlay $\Pi$ and underlying $\mathcal{Y} = \Y[][1], \ldots, \Y[][n]$,
  is a family of interactive machines
  $\Lambda_{\Pi,\mathcal{Y}}$ providing the following
  functionalities:

  \begin{enumerate}
    \item \construct$(\sid, (\Y[j][1] \ldots \Y[j][n]))$.
    \item \writeToMachine$(i, \data)$.
    \item \emulationSnapshot$(i, r)$. % <-- Z
  \end{enumerate}
\end{definition}

The compositor is constructed by calling \emph{\construct} with parameter the session identifier $\sid$.
Note that compositors are permissionless, as they don't know their instance identity $j \in \mathbb{N}$.
Each compositor instance provides a \emph{\writeToMachine}
functionality that allows writing data to the $i^\text{th}$ emulated machine.
It also provides an \emph{\emulationSnapshot} functionality that promises to emulate the execution of
the $i^\text{th}$ instance of $\Pi$ up to round $r$, and return the instance of this emulation
at round $r$ (and note that this contains the full transcript of the emulation).
Observe that \emph{\emulationSnapshot} can be invoked at a later round $r' > r$.
Note here the difference between $\mathcal{Y} = \Y[][1], \ldots, \Y[][n]$ (denoting \emph{underlying protocols}
in the form of ITMs) and
$(\Y[j][1], \ldots, \Y[j][n])$ (denoting \emph{underlying protocol instances}
in the form of ITIs). The compositor is parametrized with the former, and
the latter are passed as parameters to the \emph{\construct} functionality.

Even though we will not do the analysis in the full Universal Composability (UC) framework, and
we treat executions as stand-alone, we do adopt some of the notation of the UC framework.
Let $\LCExec(\Lambda, \mathcal{Y}, \mathcal{A}, \mathcal{Z})$
denote the transcript of an execution of the compositor protocol $\Lambda$ parametrized with overlay protocol $\Pi$,
and underlying protocols $\mathcal{Y} = (\Y[][1], \ldots, \Y[][n])$, adversary $\mathcal{A}$, and environment $\mathcal{Z}$.
Following the notation of the UC framework, we define the notion of an \emph{execution}
and a \emph{view}.

\myparagraph[The Emulated Setting]
In particular, in the emulated setting execution, denoted by $\LCExec_{m,n}(\Lambda, \mathcal{Y}, \mathcal{A}, \mathcal{Z})$,
the environment $\mathcal{Z}$ is constrained by the control program
to initially
spawn a number $n \times \ell$ (where $n$ is a parameter of the execution,
and $\ell = |\mathcal{Y}|$)
underlying protocol instances
\begin{align*}
\Y[1][1], &\ldots, \Y[1][\ell],\\
  \vdots, &\ddots, \vdots      \\
\Y[m][1], &\ldots, \Y[m][\ell]\,,
\end{align*}
where $\Y[j][i]$ denotes the $j^\text{th}$ instance of
the protocol $\Y[][i]$. The environment facilitates the communication
between the underlying protocol instances $\Y[j][i]$ and $\Y[j'][i]$
for all $j, j'$, whereas protocols $\Y[j][i]$ and $\Y[j'][i']$
with $i \neq i'$ are not allowed to directly communicate.
Next, the environment is constrained to spawn
a number $m$ of compositor clients
$\LLambda[1], \ldots, \LLambda[m]$ (where $m \in \mathbb{N}$
is a parameter of the execution), allowing
the environment to choose the $\sid$ parameter,
and passing the tuple $(\Y[j][1], \ldots, \Y[j][\ell])$ of ITIs
to $\LLambda[j]$'s \emph{construct} functionality
(in UC language, each of these $\Y[j][i]$ is constrained to be used
as a \emph{subroutine} solely by $\LLambda[j]$).
All of those $\Lambda$ compositor clients are honest and will remain
so throughout the execution.
Outside of those $\Y[j][i]$ instances controlled by the $\Lambda$s, each of
the protocols $\Y[][i]$ may have more instances running within the execution,
spawned by the environment and potentially corrupted by the adversary.
For example, if $\Y[][1]$ is the Bitcoin protocol, then the $\Y[j][i]$
instance running within $\LLambda[j]$ is a Bitcoin client, whereas the
execution comprises also others Bitcoin clients and full nodes, potentially
corrupted by the adversary.
The execution proceeds in rounds $r = 1, \ldots, R$
for a polynomial number $R$ of rounds (where the polynomial
is taken with respect to the security parameter). For every
round $r$, the environment is constrained to first call
the \emph{execute} function of each $\Y[j][i]$ for every
$j, i \in \mathbb{N}$, as well as the \emph{execute} function of $\Y$s
living outside of the clients $\Lambda$. Next, the environment must call
the \emph{execute} function of each $\LLambda[j]$ sequentially.
Finally, the environment is constrained to call the
adversary $\mathcal{Z}$ (a rushing adversary).
The adversary is allowed to corrupt the $\Y$ instances living outside of $\Lambda$s
% TODO: Can there even be a safety-respecting environment, e.g., in Streamlet or Bitcoin?
% I think the environment can only be things like honest-majority-respecting.
% How can an environment constraint the execution to be safe?
(we will later impose constraints, in the form of \emph{beliefs}, which the
environment ensures the adversary must respect).
At any time, the environment may choose to provide inputs
to any of the $\LLambda[j]$ parties by invoking their \emph{writeToMachine}
functionality with inputs of its choice.

\myparagraph[The Party Setting]
In the party setting execution, denoted by $\PEExec_n(\Pi,\allowbreak \mathcal{A},\allowbreak \mathcal{Z},\allowbreak n,\allowbreak H,\allowbreak \Delta)$,
the environment $\mathcal{Z}$ is constrained by the control program
to spawn $n$ parties $\PPi[1], \ldots, \PPi[n]$ (where $n \in \mathbb{N}$
is a parameter of the execution). Note here that $\Pi$ is a protocol (an ITM),
whereas each $\PPi[i]$ is an instance (an ITI).
The adversary $\mathcal{A}$ is allowed to corrupt parties
indexed by $[n] \setminus H$ at the beginning of the execution
(a static corruption model). This is done by the adversary sending a message
to the environment requesting to corrupt the desired party. The environment
grants this corruption wish as long as it respects the requirement that the
corruption falls within $[n] \setminus H$.
The execution proceeds in rounds
$r = 1, \ldots, R$, where $R$, again, is a polynomial of the security
parameter. At every round, the environment is constrained to first
call the \emph{execute} function of each $\PPi[i]$ for every
honest party, in order. Next, the environment must call
the adversary (again, a rushing adversary). The environment
is constrained to deliver messages between honest parties
in an authenticated manner, and to deliver messages within $\Delta$
delay.

We would like to define a notion of \emph{faithfulness} of a
compositor, which captures the
correspondence between the party setting execution and the emulated setting
execution of $\Pi$. Roughly, a compositor is called \emph{faithful} is these
two settings are identical in the eyes of the honest parties.
This faithfulness may be conditioned to work only on certain classes
of overlay protocols $\Pi$ and underlying protocols $\Y$, and may
require that a certain subset $\mathcal{H}$ of these $\Y$s are well-behaved.

\begin{definition}[Compositor Faithfulness (informal)]
We say that a compositor $\Lambda$ is \emph{faithful} for an overlay protocol $\Pi$,
a number of overlay parties $n \in \mathbb{N}$ running
over a number of underlying protocols $\mathcal{Y} = (\Y[][1], \ldots, \Y[][\ell])$
if:
For all number of compositor parties $m \in \mathbb{N}$,
for every compositor index $j \in [m]$,
for every overlay index $i \in [n]$ that ``corresponds' to ``well-behaving'' underlying $\mathcal{Y}$s
it holds that:
The party $\Z[j][i]$'s emulated execution ``is identical'' to \emph{some} party setting execution
(it cannot tell if it is emulated or not).
Within that emulated execution, the emulated instance $\Z[j][i]$
is given the ``same'' inputs as \emph{write}(\emph{data}) by its environment
as the compositor is given by its own environment as \emph{writeToMachine}(i, \emph{data})
for that same $i$.
\end{definition}

The motivation for the above definition stems from the fact that, if it is known that
$\Pi$ is secure in the party setting, these security results can be translated to the
emulated setting. The full definition of faithfulness will state that for all adversaries
in the emulated setting, there is a simulator in the party setting that makes the
views of honest parties identical. Since the protocol $\Pi$ is secure in the party setting
under that simulator, it must also be secure in the emulated setting under any adversary.
This line of argument is not unlike Canetti's UC arguments.

While the fact that the view of the honest party is the same in both the emulated and
party setting is sufficient to argue that the \emph{write} instructions in both settings will
be the same in the views of the honest parties, this will not be sufficient.
The last part of the definition sketch above whereby the \emph{writeToMachine} instructions
of the emulated setting are replicated as \emph{write} instructions in the party setting
is a necessary ingredient to make the definition useful. Otherwise, trivial constructions
in which \emph{writeToMachine} instructions are ignored are possible, yet we want to avoid
such pathologies.

To make the above definition precise, we must develop a number of tools. Namely,
we must state what ``well-behaving'' underlying protocols are, and what the ``correspondence''
between overlay parties and underlying protocols means. Furthermore, we must specify
what the ``identical'' emulated execution means, and what the ``same'' inputs are,
in which definitions we will be required to allow for some slack.

We define the following views for the two execution settings.

\begin{definition}[Honest View in the Party Setting]
  Consider a party setting execution $\mathcal{E}'$ of duration $R$ rounds
  sampled from
  $\PEExec_n(\Pi,\allowbreak \mathcal{A},\allowbreak \mathcal{Z},\allowbreak n,\allowbreak H,\allowbreak \Delta)$.
  The \emph{party setting view of honest parties} $\PEView_H(\mathcal{E}')$
  is the $|H|\times R$ matrix
  \[
  \begin{bmatrix}
    \PPi[H_1]_1 &\ldots& \PPi[H_1]_R\\
         \vdots &\ddots& \vdots     \\
    \PPi[H_{|H|}]_1 &\ldots& \PPi[H_{|H|}]_R
  \end{bmatrix}\,,
  \]
  of the transcripts of honest parties
  where $\PPi[H_i]_r$ denotes the transcript of party $\PPi[H_i]$
  (the party with index $H_i$, the $i^\text{th}$ honest party)
  obtained at the end of round $r$.
\end{definition}

Note that, in the above definition, the transcripts concerned pertain to the
set $H$ of guaranteed honest party indices only, even though the adversary may choose
to leave some of the other parties uncorrupted, too. The transcript of
those other parties are not included in $\PEView_H(\mathcal{E}')$.

Note also that, in each row of the above definition, the transcripts
are taken for a particular party $H_i$ at increasing rounds, and therefore
we will have that $\PPi[H_i]_r \preceq \PPi[H_i]_{r+1}$ and, so, the transcripts
recorded in each row will be growing in an append-only fashion.

\begin{definition}[Emulation Consistency]
  An execution $\mathcal{E}$ with duration $R$ rounds sampled from
  $\LCExec_m(\Lambda, \mathcal{Y}, \mathcal{A}, \mathcal{Z})$
  is \emph{$(j,H,\Delta_v)$-\emph{consistent}},
  for a compositor index $j \in [m]$,
  a set of overlay machine indices $H \subseteq [n]$,
  and reality lag $\Delta_v$,
  if
  for all $i \in H$, for all $r \geq 0$,
  for all $r + \Delta_v < r' < R$,
  it holds that
  $\LLambda[j].\emulationSnapshot(i, r)$ executed \emph{in vitro} at the end of round $r'$
  is equal to
  $\LLambda[j].\emulationSnapshot(i, r)$ executed \emph{in vitro} at the end of round $r' + 1$.
\end{definition}

Note that in the above definition, we allow $r = 0$, even though rounds
in the execution begin at $1$.

\begin{definition}[Honest View in the Emulated Setting]
  Consider an emulated setting execution $\mathcal{E}$ with duration $R$ rounds
  sampled from $\LCExec_m(\Lambda, \mathcal{Y}, \mathcal{A}, \mathcal{Z})$.
  If $\mathcal{E}$ is $(j,H,\Delta_v)$-consistent, then
  the \emph{emulated setting view of honest parties} $\LCView_{j,H,\Delta_v}(\mathcal{E})$,
  % TODO: change "reality lag" to be a generic "time distortion" function,
  % for now f(r) = r + \Delta_v, but this could be more generic
  parametrized by an index $j$, a set of indices $H$, and a \emph{reality lag} $\Delta_v \in \mathbb{N}$,
  is the $|H| \times R$ matrix
  \[
  \begin{bmatrix}
    E(H_1, 1) & \ldots & E(H_1, R - \Delta_v - 1)\\
       \vdots & \ddots & \vdots \\
    E(H_{|H|}, 1) & \ldots & E(H_{|H|}, R - \Delta_v - 1)
  \end{bmatrix}\,,
  \]
  where $E(H_{|H|}, 1)$ denotes the return value of invoking,
  \emph{in vitro} at the end of round $r + \Delta_v$,
  the $\emulationSnapshot$ functionality of the $j^\text{th}$ compositor party $\LLambda[j]$
  with parameters the index $H_i$ of the $i^\text{th}$ overlay machine (among those
  included in $H$) and round $r$.

  On the other hand, if the execution $\mathcal{E}$ is \emph{not} $(j,H,\Delta_v)$-consistent, then
  we let $\LCView_{j,H,\Delta_v}(\mathcal{E}) = \bot$.
\end{definition}

\begin{definition}[Party Setting Externalities]
  Consider a party setting view $V$ of honest parties and size $|H| \times R$.
  The \emph{party setting externalities} $\PEExtern(V)$ is the $|H| \times R$ matrix
  \[
  \begin{bmatrix}
    \W[H_1][1] &\ldots& \W[H_1][R]\\
         \vdots &\ddots& \vdots     \\
    \W[H_{|H|}][1] &\ldots& \W[H_{|H|}][R]
  \end{bmatrix}\,,
  \]
  where $\W[H_i][r]$ denotes the sequence of messages written into an honest party
  with index $H_i$ during round $r$. This sequence of messages can be extracted from
  the transcript $\PPi[H_i]_r$ found in $V$.
\end{definition}

\begin{definition}[Emulated Setting Externalities]
  Consider an emulated setting execution $\mathcal{E}$ with duration $R$ rounds
  sampled from $\LCExec_m(\Lambda,\allowbreak \mathcal{Y},\allowbreak \mathcal{A},\allowbreak \mathcal{Z})$.
  Consider the transcript of compositor client $\LLambda[j]$ in $\mathcal{E}$.
  Within that transcript, observe the \emph{writeToMachine} calls made by
  $\mathcal{Z}$ on $\LLambda[j]$ during some fixed round $1 \leq r \leq R$.
  Among those, consider the calls to \emph{writeToMachine} that were
  invoked with first argument some fixed machine index $1 \leq i \leq m$.
  These \emph{writeToMachine} calls were invoked, in order, as
  $\writeToMachine(i, \data_1), \ldots, \writeToMachine(i, \data_k)$
  all during round $r$.
  Let $\W[i][r] = (\data_1,\allowbreak\ldots,\allowbreak\data_k)$ denote the sequence containing
  all the \emph{data} parameter values of those calls.
  The \emph{emulated setting externalities} $\LCExtern_{j,H}(\mathcal{E})$, parametrized
  by an index $j$ and a set of indices $H$, is the $|H| \times R$ matrix
  \[
  \begin{bmatrix}
    \W[H_1][1] &\ldots& \W[H_1][R]\\
         \vdots &\ddots& \vdots     \\
    \W[H_{|H|}][1] &\ldots& \W[H_{|H|}][R]
  \end{bmatrix}\,.
  \]
\end{definition}

\begin{definition}[Externality Similarity]
  Consider the externalities $E_1, E_2$ (in the party or emulated setting)
  with dimensions $n \times R_1$ and $n \times R_2$ respectively.
  We say that $E_1$ is \emph{similar} to $E_2$ with \emph{lateness} parameter
  $\Delta_u \in \mathbb{N}$ and \emph{earliness} parameter $\Delta_v \in \mathbb{N}$, written
  $E_1 \laterly{\Delta_u,\Delta_v} E_2$, if
  the following holds:
  For any message $m$ located within the writebox at position $(i, r)$ of $E_1$,
  with $r < R_1 - \Delta_u - \Delta_v$,
  there exists a round $r - \Delta_v \leq r' \leq r + \Delta_u$ such that
  the message $m$ appears within the writebox at position $(i, r')$ of $E_2$.
\end{definition}

% TODO: Simplify the below; first define the "distance" between two externalities
% E_1, E_2 to create a metric space of externalities.
% The distance will be defined as the maximum delay of messages
% between two externalities.
% Afterwards, we can define the distance between the two externalities
% random variables using the Wasserstein metric on the distance between
% the externalities.

\begin{definition}[Externality Similarity in Distribution]
  Consider the externalities random variables $E_1, E_2$.
  We say that $E_1$ is \emph{similar in distribution} to $E_2$
  with \emph{lateness} parameter $\Delta_u \in \mathbb{N}$
  and \emph{earliness} parameter $\Delta_v \in \mathbb{N}$,
  written $E_1 \laterlydistr{\Delta_u,\Delta_v} E_2$, if
  there exists a sample space $\Omega$ and two coupled
  random variables $\widetilde{E}_1(\omega), \widetilde{E}_2(\omega)$
  such that $E_1' \distreq E_1$ and $E_2' \distreq E_2$
  and $E_1 \laterly{\Delta_u,\Delta_v} E_2$.
\end{definition}

\begin{definition}[Belief]
  A \emph{belief} $B$ is any predicate over an execution $\mathcal{E}$.
\end{definition}

For example, given an execution $\mathcal{E}$, with underlying distributed ledger
protocols $\mathcal{Y} = (\Y[][1], \ldots, \Y[][\ell])$, we can define a belief
$B$ asserting that the majority of underlying ledger protocols are secure.

\begin{definition}[Belief System]
  A \emph{belief system} $\mathcal{B}$ is a set of beliefs.
\end{definition}

\begin{definition}[Honesty Correspondence]
  Given a belief system $\mathcal{B}$ and a number $n \in \mathbb{N}$ of overlay parties,
  an \emph{honesty correspondence} $\mathcal{H}(B)$ is any function
  from $\mathcal{B} \longrightarrow 2^{[n]}$.
\end{definition}

\begin{definition}[Belief-Respecting Environment]
  An environment $\mathcal{Z}$ is \emph{belief-respecting} for a belief $B$
  if for all executions $\mathcal{E}$ in the support of
  a given distribution of executions, it holds that $B(\mathcal{E})$.
\end{definition}

% TODO: Mention that the validator and other protocol settings for \Y[][i]
% are hard-coded into \Y[][i]
\begin{definition}[Emulation Faithfulness]\label{def:emulation-faithfulness}
  A compositor $\Lambda$
  is $(\Pi,\allowbreak n,\allowbreak \mathcal{Y},\allowbreak \mathcal{B},\allowbreak \mathcal{H},\allowbreak \Delta_u,\allowbreak \Delta_v)$-\emph{emulation-faithful}
  for an overlay protocol $\Pi$,
  a number of overlay parties $n \in \mathbb{N}$,
  a sequence of underlying protocols $\mathcal{Y} = (\Y[][1], \ldots, \Y[][\ell])$,
  a belief system $\mathcal{B}$,
  honesty correspondence $\mathcal{H}: \mathcal{B} \longrightarrow 2^{[n]}$,
  lateness $\Delta_u \in \mathbb{N}$,
  and reality lag $\Delta_v \in \mathbb{N}$
  if:

  For all beliefs $B \in \mathcal{B}$,
  for all PPT adversaries $\mathcal{A}$ and all
  $B$-respecting PPT environments $\mathcal{Z}$,
  for all number of compositor parties $m \in \mathbb{N}$,
  for all compositor party indices $j \in [m]$,
  % TODO: There is no need for the polynomiality constraint on A and Z,
  % but the reduction maintains polynomiality
  there is a PPT simulator $\mathcal{S}$ and there is a
  PPT environment $\mathcal{Z}'$ such that
  the following holds:

  \begin{enumerate}
    \item $\LCView_{j,\mathcal{H}(B),\Delta_v}(\mathcal{E}) \distreq \PEView_{\mathcal{H}(B)}(\mathcal{E}')$
    \item $\LCExtern_{j,\mathcal{H}(B)}(\mathcal{E}) \laterlydistr{\Delta_u,\Delta_v} \PEExtern(\PEView_{\mathcal{H}(B)}(\mathcal{E}'))$
  \end{enumerate}

  where execution $\mathcal{E}$ is sampled from
  % TODO: Do not use "EXEC", as this is inconsistent with the UC framework
  $\LCExec_m(\Lambda, \mathcal{Y}, \mathcal{A}, \mathcal{Z})$
  % TODO: B should only depend on the collective execution of \Y, not anything else
  and execution $\mathcal{E}'$ is sampled from
  $\PEExec(\Pi, \mathcal{S}, \mathcal{Z}', n, \mathcal{H}(B), \Delta)$,
  and $\Delta = 2 \Delta_v + \Delta_u$.
\end{definition}

% Remark: Differences from a UC-style proof

% TODO: DRY from Emulation Faithfulness
\begin{definition}[Replication Faithfulness]\label{def:replication-faithfulness}
  A compositor $\Lambda$ is
  $(\Pi, n, \mathcal{Y}, \mathcal{B}, \mathcal{H}, \Delta_u, \Delta_v)$-\emph{replication-faithful}
  for overlay protocol $\Pi$,
  number of overlay parties $n \in \mathbb{N}$,
  a sequence of underlying protocols $\mathcal{Y} = (\Y[][1], \ldots, \Y[][\ell])$,
  belief system $\mathcal{B}$,
  honesty correspondence $\mathcal{H}: \mathcal{B} \longrightarrow 2^{[n]}$,
  lateness $\Delta_u \in \mathbb{N}$,
  and reality lag $\Delta_v \in \mathbb{N}$
  if:

  For all beliefs $B \in \mathcal{B}$,
  all all PPT adversaries $\mathcal{A}$,
  for all $B$-respecting PPT environments $\mathcal{Z}$,
  for all number of compositor parties $m \in \mathbb{N}$,
  for all compositor party indices $j, j' \in [m]$
  and for all overlay party indices $i \in \mathcal{H}(B)$,
  for all rounds $1 \leq r < R - \max(\Delta_v, \Delta_u)$,
  and all compositor party indexes $j, j' \in [m]$ it holds that
  $\Sim[j][r] = \Sim[j'][r]$,
  where $\Sim[j][r]$ (resp. $\Sim[j'][r]$) indicates the result of calling
  \emulationSnapshot of $\Lambda[j]$ (resp. $\Lambda[j']$) with inputs
  $(i, r)$
  \emph{in vitro} at the end of round $r + \Delta_v$ of $\mathcal{E}$,
  where $\mathcal{E}$ is an execution sampled from
  $\LCExec_m(\Lambda, \mathcal{Y}, \mathcal{A}, \mathcal{Z})$.
\end{definition}
