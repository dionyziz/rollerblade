\section{Definitions}

\subsection{The Setting}

We are given $n \in \mathbb{N}$ ledger protocols
$\Y[][1], \Y[][2], \ldots, \Y[][i], \ldots, \Y[][n]$,
the so-called \emph{underlying} ledger protocols.
While mathematically, these $\Y$s are interactive Turing machines of ledger protocols,
in practice, these are preexisting, already operational ledger protocol executions
such as Bitcoin, Ethereum, and Cardano, for which we have access to already running full nodes
and we are asked to compose on top of.

We are also given a distributed protocol $\Pi$ (not necessarily a ledger protocol),
the so-called \emph{overlay} protocol. We will simulate an $n$-party execution of $\Pi$,
with each $i$ of these $n$ parties corresponding to the underlying ledger protocol $\Y[][i]$.

The users of the protocol are $m \in \mathbb{N}$ \rollerblade \emph{clients} termed
$\RB[1], \RB[2], \ldots, \RB[j], \ldots, \RB[m]$ (with, potentially $m \neq n$).
\emph{Each} $\RB[j]$ client runs a separate full node $\Y[j][1], \ldots, \Y[j][i], \ldots, \Y[j][n]$
for each of the underlying $\Y$s. The $\RB$ nodes do not have direct network communication, but only
use the read/write functionalities of their respective $\Y$ instances to communicate.
For example, when party $\RB[1]$ \emph{writes} a transaction $\tx$ to its $\Y[1][1]$ instance,
this transaction will eventually appear in $\RB[2]$'s $\Y[2][1]$ instance ledger output,
as long as $\Y[][1]$ is live.
We will use $\Ledger[j][i][r] \gets \Y[j][i][r].\lread()$ to refer to the ledger reported at
round $r$ by the full node instance $\Y[j][i]$ running the underlying ledger protocol $\Y[][i]$
operated by the overlay party $\RB[j]$
(this ledger, like all ledgers, is a sequence of round/transaction pairs, and its
$k^\text{th}$ round/transaction pair is $\Ledger[j][i][r][k]$).

%
% TODO: move this to consensus section
% Our goal is to build a \emph{new} DLP $\RB$
% on top of these such that, if the majority of \emph{underlying}
% $\wheel[1], \ldots, \wheel[n]$ are \emph{secure}, then so is the
% \emph{overlay} DLP $\RB$. As $\RB$ is a DLP, it will have its own population of $m$ nodes
% $\RB_1, \RB_2, \ldots, \RB_m$. The $\RB$ DLP will have its own type of transaction and a
% ledger consisting of these transactions. For example, it can maintain its own coin\footnote{
% We will not be concerned with bridging this coin from the overlay ledger to the underlying
% ledgers. This can be done using various standard bridging techniques which are orthogonal
% to this work~\cite{pow-sidechains,pos-sidechains,zkbridge}.}. Importantly, similarly to
% a rollup, \emph{we will not make any additional security assumptions about the honesty of the
% $\RB$ nodes}.
%

We now describe the requirements of our underlying and overlay protocols.

\subsection{Underlying Requirements}

Firstly, in order to record data on our underlying ledgers, we require that any arbitrary string
can be written to them. This is called a \emph{bulletin}.

\paragraph{Bulletins}
% We do not want to modify the underlying DLPs to support our protocol. However, the underlying DLPs
% do not understand the transaction semantics of our overlay ledger. We will therefore use the underlying
% DLPs as a simple lazy transaction ordering service. We will assume that we can take any
% arbitrary string $w$ and \emph{encode} it into a transaction suitable for each underlying ledger.
% Such transactions are \emph{always} accepted by the underlying ledger, and their contents
% are not validated (beyond checking that a minimum fee is suitably paid to avoid spamming attacks).
% We call a DLP that supports writing such arbitrary data into it a \emph{bulletin}.
A bulletin ledger protocol offers two additional
functions $\encode$ and $\decode$:
$\tx \gets \textsf{encode}(s)$ and $s \gets \textsf{decode}(\tx)$.
The $\textsf{encode}$ function takes a string $s$ and encodes it into a transaction $\tx$ that can be \emph{written}
into the ledger and is guaranteed to be accepted.
The $\textsf{decode}$ function takes a transaction $\tx$
and, if it is a bulletin transaction, decodes it back into $s$.
Otherwise, $\textsf{decode}$ can return $\bot$.
All transactions produced by $\textsf{encode}$
are bulletin transactions, but the adversary can also introduce arbitrary bulletin transactions of her choice
indiscriminately. The ledger may also include non-bulletin transactions among the bulletin transactions.

\begin{definition}[Bulletin Board]
  A ledger protocol $\Pi$ accompanied by a pair of computable
  functions $(\encode, \decode)$, of which $\decode$ is deterministic,
  is called a \emph{bulletin board} if it holds that, for any $s \in \{0, 1\}^*$,
  the output of $\tx = \encode(s)$ is always a valid
  transaction and that $\decode(\encode(s)) = s$.
\end{definition}

Bulletins provide ordering and data availability of arbitrary data without checking
any semantic validity. As such, they constitute a \emph{lazy} use of a ledger~\cite{lazyledger,lazylight}.
All popular blockchains such as, for example, Ethereum and Bitcoin are bulletins.
Bitcoin allows the recording of arbitrary data using \textsf{OP\_RETURN}
transactions, whereas Ethereum allows such recording by including the
data in the \textsf{CALLDATA} of a smart contract call, or in the parameters
of an event.

\paragraph{Certifiability}

\begin{definition}[Certifiability]
  A ledger protocol $\Pi$ accompanied by a functionality $\transcribe$
  and a computable deterministic function $\untranscribe$ is called
  \emph{certifiable} if, ???
  % TODO: describe safety and liveness in context of certificates
\end{definition}

% TODO: restore this definition
% \begin{definition}[Transcribability]
%       Definition of transcribability...
% \end{definition}
%