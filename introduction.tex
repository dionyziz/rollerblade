\section{Introduction}
So many blockchains are out there nowadays. Which one should we choose to secure
our money? All choices might seem wise today, but any may fail tomorrow.

In this paper, we propose \emph{Rollerblade}: A ledger construction
that runs on top of other chains. If the
\emph{majority} of the \emph{underlying} chains remains secure, the rollerblade running
\emph{on top} will also be secure.
We are making a statement of \emph{reliability}: We are \emph{composing}
ledger protocols and building a new ledger protocol that enjoys
security even if some of them are faulty.

Our construction can use any distributed ledger protocol (DLP) as the underlying
composed protocols. These can be proof-of-work, proof-of-stake, or permissioned
protocols, or a combination thereof.

\noindent
\textbf{Our contributions.} The contributions of this paper are the following:

\begin{enumerate}
  \item We put forth \emph{rollerblade}, a ledger composition protocol that achieves
        reliability.
  \item We prove that rollerblade enjoys security if the underlying protocols
        are secure by majority.
  \item We axiomatize our composition so that \emph{any} proof-of-work, proof-of-stake,
        permissioned protocol, or combination thereof, can be used as foundation,
        and any BFT protocol can run on top as the overlay protocol to compose them.
\end{enumerate}

\noindent
\textbf{Related work.} Building a reliable system by composing potentially faulty components
is a classical engineering problem~\cite{von1956probabilistic,moore1956reliable}.
In the era of consensus, the seminal
paper by Lamport that introduced the Byzantine Fault Tolerance problem~\cite{shostak1982byzantine}
aims to solve a reliability problem, where different processors disagree about their
outcomes. The composition of multiple \emph{blockchain} protocols was explored by
Kiayias et al.~cite{fitzi2020ledger}, but for the purpose of performance in terms of throughput
and latency, not reliability. The idea of borrowing security from one chain to
secure another was first proposed in the context of \emph{merged mining}~\cite{namecoin}, where
a strong proof-of-work blockchain secures a weaker proof-of-work blockchain. This
concept was later extended to \emph{merged staking}~\cite{pos-sidechains}, where a
strong proof-of-stake \emph{parent chain} lends its security to a less secure
\emph{child chain}. Similar ideas have recently appeared in Cosmos proof-of-stake
chains as \emph{Mesh Security}. Hybrids between different consensus mechanisms
have also been proposed. The idea of \emph{checkpointing}
one blockchain onto another to borrow security was formalized by Karakostas et al.~\cite{karakostas2021securing},
and has been applied to secure newly created proof-of-work systems using stronger
preexisting proof-of-stake systems. Such constructions are uni-directional:
one ``parent'' chain is designated as authoritative and \emph{stronger},
while the ``child'' chain is designated as \emph{weaker} and in need of securing.
The idea of composing ledgers to achieve a more reliable overlay ledger
was first proposed in a short Cosmos GitHub issue called
\emph{recursive Tendermint}~\cite{recursive-tendermint}.
This concept was expanded upon by TrustBoost~\cite{trustboost}
where they build a Cosmos-specific construction of ledger composition
using IBC for communication and particular to BFT-style protocols.

\noindent
\textbf{Construction overview.}
The core idea is to checkpoint all underlying DLPs among each other, in
a way that is symmetric and can survive the demise of any minority of the
DLPs.
In a nutshell, rollerblade works by executing any permissioned BFT protocol,
such as Streamlet or HotStuff in a \emph{virtual execution}. This execution is
conducted by the users interested in the protocol (such as the economic stakeholders
in the protocol), who facilitate the protocol's execution by reading and interpreting
the rollerblade-specific transactions written on the underlying ledgers, as well as
posting transactions on the underlying ledgers that cross-checkpoint .

in . At every moment, it is
assumed that at least one honest rollerblade user is online (the
\emph{existential honesty} assumption~\cite{backbone}).
the underlying DLPs are treated as parties.
