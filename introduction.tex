\section{Introduction}
So many blockchains are out there nowadays. Which one should we choose to secure
our money? All choices might seem wise today, but any may fail tomorrow.

In this paper, we propose \emph{Rollerblade}: A ledger construction
that runs on top of other chains. If the
\emph{majority} of the \emph{underlying} chains remains secure, the rollerblade running
\emph{on top} will also be secure.
We are making a statement of \emph{reliability}: We are \emph{composing}
ledger protocols and building a new ledger protocol that enjoys
security even if some of them are faulty.

Our construction can use any distributed ledger protocol (DLP) as the underlying
composed protocols. These can be proof-of-work, proof-of-stake, or permissioned
protocols, or a combination thereof.

\noindent
\textbf{Our contributions.} The contributions of this paper are the following:

\begin{enumerate}
  \item We put forth \emph{rollerblade}, a ledger composition protocol that achieves
        reliability.
  \item We prove that rollerblade enjoys security if the underlying protocols
        are secure by majority.
  \item We axiomatize our composition so that \emph{any} proof-of-work, proof-of-stake,
        permissioned protocol, or combination thereof, can be used as foundation,
        and any BFT protocol can run on top as the overlay protocol to compose them.
\end{enumerate}

\noindent
\textbf{Related work.} Building a reliable system by composing potentially faulty components
is a classical engineering problem~\cite{von1956probabilistic,moore1956reliable}.
In the era of consensus, the seminal
paper by Lamport that introduced the Byzantine Fault Tolerance problem~\cite{shostak1982byzantine}
aims to solve a reliability problem, where different processors disagree about their
outcomes. The composition of multiple \emph{blockchain} protocols was explored by
Kiayias et al.~cite{fitzi2020ledger}, but for the purpose of performance in terms of throughput
and latency, not reliability. The idea of borrowing security from one chain to
secure another was first proposed in the context of \emph{merged mining}~\cite{namecoin}, where
a strong proof-of-work blockchain secures a weaker proof-of-work blockchain. This
concept was later extended to \emph{merged staking}~\cite{pos-sidechains}, where a
strong proof-of-stake \emph{parent chain} lends its security to a less secure
\emph{child chain}. Similar ideas have recently appeared in Cosmos proof-of-stake
chains as \emph{Mesh Security}. Hybrids between different consensus mechanisms
have also been proposed. The idea of \emph{checkpointing}
one blockchain onto another to borrow security was formalized by Karakostas et al.~\cite{karakostas2021securing},
and has been applied to secure newly created proof-of-work systems using stronger
preexisting proof-of-stake systems. Such constructions are uni-directional:
one ``parent'' chain is designated as authoritative and \emph{stronger},
while the ``child'' chain is designated as \emph{weaker} and in need of securing.
The idea of composing ledgers to achieve a more reliable overlay ledger
was first proposed in a short Cosmos GitHub issue called
\emph{recursive Tendermint}~\cite{recursive-tendermint}.
This concept was expanded upon by TrustBoost~\cite{trustboost}
where they build a Cosmos-specific construction of ledger composition
using IBC for communication and particular to BFT-style protocols.

\noindent
\textbf{Practical efficiency.}
Our construction is about what compositions of ledgers are theoretically \emph{possible},
so we will not be concerned with the efficiency of our construction beyond the
theoretical desire to remain polynomial. Our treatment of
ledgers is generic, and our aim is to formulate a minimal set of axioms
required of underlying and overlay ledger protocols to render them composable.
Due to the generality of our construction, certain optimizations will not be
possible, but efficiency can be greatly increased for concrete DLPs.
For a particular example on Tendermint, see TrustBoost~\cite{trustboost}.

\noindent
\textbf{Construction overview and paper structure.}
The execution is conducted by the overlay users of the protocol.
Each of them operates a full node for each of the underlying DLP.
When any of the overlay nodes wishes to write a transaction for
their ledger, they replicate and post that transaction to each of
the underlying DLPs. They then wait for the transaction to be included
in each of the underlying DLPs, then take a majority vote among them
to determine their own reported ledger. This construction is described
in Section~\ref{sec:construction-naive}.

Unfortunately, a simple majority vote is insufficient because each
underlying DLP may include transactions in a different order. Hence,
the disagreement must be solved in a fashion akin to running a BFT
(such as \emph{Streamlet}~\cite{streamlet} or \emph{HotStuff}~\cite{hotstuff})
protocol ``on top''. The core idea is to treat every underlying DLP
as a \emph{participant} in the overlay BFT. The \emph{security} of
each DLP then roughly corresponds to the \emph{honesty} of the
respective party. Due to the security of the overlay BFT protocol
and the fact that the majority of underlying DLPs are secure, the
resulting ledger read from the overlay BFT will allow the overlay
parties to reach agreement between themselves. This construction
is described in Section~\ref{sec:construction-bft}.

BFT protocols are designed to run between \emph{parties}, not DLPs.
A translation of the BFT protocol must take place so that it can be
utilized with DLPs as parties. This requires some technical work
which entails swapping out any of its cryptographic signature creation
and verification, and the removal of any non-determinism by delegating
it to an oracle. The networking assumptions of the BFT protocol then
must be translated to the respective DLP assumptions by treating
network delay as ledger liveness. These \emph{virtualization} technical
details are given in Section~\ref{sec:construction-oraclized}.

Next, we show that our construction applies to a wide variety of
protocols by discussing how it can be run on top of proof-of-work,
proof-of-stake, and permissioned protocols. We also remove the requirement
that any of the underlying chains require any smart contract capability,
although smart contracts can improve our performance. These compatibility
issues are discussed and the exact mechanism of mutual checkpointing
that does not rely on smart contracts is discussed in
Section~\ref{sec:compatibility}.

In Section~\ref{sec:analysis} we prove that our protocol is secure by
stating and showing a \emph{metatheorem} which, in essence, tells us that any
party-based execution of a BFT protocol and the respective DLP-based
execution of the same protocol suitably translated are the same. This
then allows the translation of any existing BFT theorem (such as Streamlet's
or HotStuff's security theorems) from the party setting to the DLP setting,
treating both the BFT protocol on top as well as the underlying DLPs as
black boxes following a minimal set of axioms.
