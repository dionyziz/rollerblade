\appsection[A Consensus Application]\label{sec:consensus}

One way to run \rollerblade is with an overlay consensus protocol.
In this construction, each overlay party acts as a validator.
Since each overlay party depends on the security of its
corresponding underlying ledger, the security resilience
of the overlay protocol now dictates the number of well-behaved
underlying ledgers required.
If $\Pi$ has $\frac{1}{2}$ resilience and the majority of underlying
ledgers are secure, then the emulated execution of $\Pi$ is also secure.
This gives life to a new kind of roll-up, which stands on top of multiple
layer-1s, making it more secure than any of them individually.

The construction is illustrated in Algorithm~\ref{alg.consensus}. The
overlay consensus client uses the \rollerblade compositor internally.
In order to \emph{write} a transaction to the overlay protocol, the
transaction is \emph{written} to every overlay party using \writeToMachine
(which, in turn, invokes the \emph{write} functionality of the underlying ledger).
There is only one non-trivial step in this construction, which pertains
to reading the current ledger. The \emph{read} functionality
of $\LLambda[j]$ must return a ledger. To do so, it requests an emulation snapshot from
each of the machines $\PPi[1], \ldots, \PPi[n]$. It then invokes \emph{read}
in each of them, obtaining ledgers $\overline{L} = (\Ledger[j][1], \ldots, \Ledger[j][n])$.
Under the assumption that the majority of the underlying ledgers are secure,
and as long as the adversarial resilience of $\Pi$ is respected, the
majority among $\overline{L}$ correspond to the ledgers reported by the
honest parties of $\Pi$'s emulated execution, and, by the security guarantees
of $\Pi$, will be safe and live. All that remains is to take a majority vote
among those ledgers. This is illustrated in Algorithm~\ref{alg.majorityvote}.

\import{./}{algorithms/alg-consensus}
\import{./}{algorithms/alg-majorityvote}

% TODO: formal proof of this
\begin{theorem}[Security]
  Let $\Pi$ be a distributed ledger protocol, and $\frac{t}{n}$ be its resilience, such
  that, if $f < t$, where $f$ is the number of adversarial parties, and $n$ is the total number of parties,
  then for all PPT adversaries, an execution of $\Pi$ is secure with overwhelming probability.

  Consider an execution of $\recursiveConsensus_{\Pi}(\mathcal{Y})$ in which $f$ of the $n$ underlying
  ledgers $\mathcal{Y}$ are secure. If $f < \min(t, \frac{n}{2})$, then the execution of $\recursiveConsensus_{\Pi}$
  is secure with overwhelming probability.
  %TODO: continue
\end{theorem}
\begin{proof}[Sketch]
  Let $\mathcal{E}$ be the execution of $\recursiveConsensus_{\Pi}$. By assumption, $f$ of
  the underlying ledgers are secure, and let $H \subseteq [n]$ be the set of indices of those underlying ledgers.

  The proof follows directly by translating the security theorem of $\Pi$ through
  the Theorem~\ref{conj:emulation}.
  \Qed
\end{proof}

% \begin{corollary}[Safety]
%       Let $\mathcal{L}$ be a weighted set of transcribable Distributed Ledger Protocols,
%       and $\Pi$ be a soft-deterministic State Machine Replication protocol.
%       Consider a rollerblade constructed with $\Pi$ as a blade and
%       $\mathcal{L}$ as its wheels.
%       Consider a rollerblade execution in which the weighted majority of $\mathcal{L}$
%       is secure.
%       Then the rollerblade is safe.
% \end{corollary}
% \begin{proof}
% \end{proof}
%
% \begin{corollary}[Liveness]
%       Let $\mathcal{L}$ be a weighted set of transcribable Distributed Ledger Protocols,
%       and $\Pi$ be a soft-deterministic State Machine Replication protocol.
%       Consider a rollerblade constructed with $\Pi$ as a blade and
%       $\mathcal{L}$ as its wheels.
%       Consider a rollerblade execution in which the weighted majority of $\mathcal{L}$
%       is secure.
%       Then the rollerblade is live.
% \end{corollary}
% \begin{proof}
% \end{proof}
