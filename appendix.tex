\section{Bitcoin is Timely}

We now prove that Bitcoin is timely. We work in the Bitcoin Backbone model~\cite{backbone}
in the synchronous setting.

First, we have to make a couple of small changes to the Bitcoin Backbone construction to turn it
into a Temporal Ledger, as the original Bitcoin Backbone construction is a ledger protocol only.
Our Temporal Bitcoin Backbone construction is illustrated in Figure~\ref{fig.temporal-backbone}.
It is clear that the Temporal Bitcoin Backbone construction retains the properties of Common Prefix,
Chain Quality and Chain Growth proven in the original Bitcoin Backbone paper~\cite{backbone}.
The reason is that the only change we make is to add a round number to the block format, and
the validity predicate of an honestly produced block is unaffected by the new validation rules.
Therefore the temporal ledger reported by the Bitcoin Backbone protocol is safe and live.

\begin{figure}
  % TODO(dionyziz): Typeset this

  The Temporal Bitcoin Backbone protocol is the same as the Bitcoin Backbone protocol, except
  for the following changes:

  \begin{enumerate}
    \item The block format is changed to include a round number and looks like this:

    \[
      B = (s, \textsf{ctr}, x, r)
    \]

    where $s$ is the hash of the previous block, \textsf{ctr} is the nonce, $x$ is the
    payload of transactions, and $r$ is a round number.

    \item The genesis block is augmented with the round $0$.
    \item The honest parties mine blocks that contain the current round number.
    \item Whenever a chain is received by an honest party, we add the following rules to
          its validation process:

          \begin{enumerate}
            \item Ensure that the rounds recorded on the received chain are strictly
                  increasing.
            \item Ensure that the round recorded on the tip is less than the current round.
          \end{enumerate}
    \item When the ledger is \textsf{read}, the reported transactions are read from the
          stable part of the chain $\chain[{:}{-k}]$. Each transaction is reported with the
          round of the block in which it is included.
  \end{enumerate}
  \caption{Temporal Bitcoin Backbone construction.}
  \label{fig.temporal-backbone}
\end{figure}

Let $\chain^P_r$ be the chain adopted by an honest party $P$ at the
beginning of round $r$ (as returned by the function \emph{maxvalid}).
For the notation definitions ($k, \ell, \tau, \mu$)
in the following theorem, please refer to the original Bitcoin
Backbone paper.

\begin{theorem}[Bitcoin Timeliness]
  A typical execution of the Temporal Bitcoin Backbone protocol is timely
  with timeliness $v = \max(s, \frac{k + \ell}{\tau})$.
\end{theorem}
\begin{proof}
  Requirements (1) and (2) of timeliness are satisfied due to the new chain validity rules.
  We will prove (3).

  Let $P$ be any honest party and $r_1 \leq r_2 \in \mathbb{N}$ be any rounds, and consider
  the ledgers $L^P_{r_1}, L^P_{r_2}$ reported by $P$ at the end of rounds $r_1, r_2$ respectively.
  Suppose, towards a contradiction, that $L^P_{r_2}[|L^P_{r_1}|{:}]$ contains a transaction
  $\tx$ with recorded round $r < r_1 - v$.

  % TODO(dionyziz): figure
  Let $\chain_1, \chain_2$ be the chains of $P$ after the conclusion of rounds $r_1$
  and $r_2$ respectively.
  Due to Common Prefix, $\chain_2$ will extend a block in $\chain_1[{-k-2}{:}]$.
  The transaction $\tx$ is contained in some block $B$ of $\chain_2[{|\chain_1[{:}{-k-1}]|}{:}]$.

  Let $B^*$ be the most recent
  honestly generated block in $\chain_1[{:}{-k-1}][-\ell{:}]$
  (or let $B^*$ be the genesis block if $|\chain_1| < \ell + k + 1$).
  This block will exist by
  Chain Quality because we are looking at a chain chunk of length at least $\ell$ and
  $\mu\ell \geq 1$.
  Block $B^*$ is honestly generated, so let $r^*$ be the round
  during which $B^*$ was generated, noting that the round recorded in $B^*$ is $r^*$.
  Let $P^*$ be the party who mined $B^*$ at round $r^*$ (or $P^* = P, r^* = 0$ if $B^*$ is
  the genesis block).

  The block $B$ extends a chain that contains $B^*$, so $r > r^*$,
  therefore $r^* < r_1 - v - 1$.

  Let $\chain^{P^*}_{r^*}$ be the chain that $B^*$ extends (this will be the empty chain if
  $B^*$ is the genesis block). If $B^*$ is not genesis, then $\chain^{P^*}_{r^*}$ is the chain
  that $P^*$ adopts in round $r^*$. Let $\chain^P_{r^* + 1}$ be the chain that $P$ adopts
  at the beginning of round $r^* + 1$. We observe

  \[
     |\chain_1| - |\chain^P_{r^* + 1}| \leq
     |\chain_1| - |\chain^{P^*}_{r^*}| - 1 \leq
     k + \ell\,.
  \]

  The first inequality stems from the fact that, at the end of round
  $r^*$, the honest party $P^*$ will broadcast the newly mined block $B^*$
  and this block will be received by $P$ at the beginning of round $r^* + 1$.
  Due to the longest chain rule, the party $P$ at the beginning of round $r^* + 1$ will
  adopt a chain of the same or larger length than
  $|\chain^{P^*}_{r^*}| + |B^*| = |\chain^{P^*}_{r^*}| + 1$,
  hence
  $|\chain^P_{r^* + 1}| \geq |\chain^{P^*}_{r^*}| + 1$.
  For the second inequality, observe that
  $\chain_1$ is the chain of $P$ after the conclusion
  of round $r_1$, whereas $\chain^{P^*}_{r^*}$ is
  the parent chain of $B^*$ and those are spaced at most $k + \ell + 1$ blocks apart by the definition
  of $B^*$.

  On the other hand, because $r_1 - (r^* + 1) > v \geq s$ we can apply Chain Growth
  between rounds $r^* + 1$ and $r_1$
  with parameters
  $s, \tau$ to obtain
  $|\chain_1| - |\chain^P_{r^*}| \geq \tau(r_1 - r^* - 1) > \tau v \geq
  \tau (\frac{k + \ell}{\tau}) \geq k + \ell$,
  which is a contradiction.

  \Qed
\end{proof}